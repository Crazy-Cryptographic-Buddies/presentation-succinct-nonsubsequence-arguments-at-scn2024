% !TEX root = main.tex
\section{Building Blocks}
\begin{frame}{Multivariate Polynomial Commitment}
	A polynomial commitment scheme (PCS, \cite{AC:KatZavGol10}) is a tuple $$(\Setup, \Com, \Open, \Eval).$$
	\begin{itemize}
		\item $ \Setup(1^\lambda)\to\pp$
		\item $\Com(\pp, f(\vec{X}))\to(\sigma(f), \aux)$
		\item $\Open(\pp, \sigma(f), f(\vec{X}), \aux)\to b$
		\item $ \Eval \innerproduct {\Prover( f(\vec{X}), \aux )} {\Verifier} (\pp, \sigma(f), \br, e)\to b$
	\end{itemize}
	
	\textbf{Properties.}
	\begin{itemize}
		\item $\sigma(f)$ is succinct. 
		\item Completeness.
		\item  Binding.
		\item $\Eval$ is knowledge-sound.
	\end{itemize}
\end{frame}
\begin{frame}{Multivariate Sumcheck}
	Given $S \in \FF$ and $f \in \FF[\mu]$ as $\mu$-variate polynomial, multivariate sumcheck \cite{FOCS:LundFKN90} allows one to prove that 
	\begin{equation*}
		\sum_{\bi \in \{0,1\}^\mu} f(\bi) = S
	\end{equation*} 
	in $\mu$ rounds where, in each row, Prover only sends a univariate polynomial to Verifier to check.
	
	The above process reduces to checking a single evaluation $f(\br)$ where $\br = (r_1, \dots, r_\mu) \in \FF^\mu$ is the vector of challenges produced by Verifier among the $\mu$ rounds.
	
	Challenges from Verifier are public coins while the final polynomial evaluation $f(\br)$ can be realized by evaluating from PCS.
	
	$\Rightarrow$ Non-interactive via Fiat-Shamir transform \cite{CRYPTO:FiaSha86} and succinct.
\end{frame}

\begin{frame}{Lookup Arguments from Multivariate Sumcheck}
	\begin{lemma}[Hab{\"o}ck \cite{EPRINT:Habock22b}]
		$\{a_j\}_{j \in [\na]} \subseteq \{b_i\}_{i \in [\nb]}$ iff one can determine $(m_i)_{i \in [\nb]}$ s.t.
		\begin{equation*}
			\sum_{j \in [\na]}\frac{1}{X - a_j} = \sum_{i \in [\nb]}\frac{m_i}{X - b_i}.
		\end{equation*}
	\end{lemma}
	
	Proving lookup by probablistically testing $\sum_{j \in [\na]}\frac{1}{\chi - a_j} = \sum_{i \in [\nb]}\frac{m_i}{\chi - b_i}$ from some random $\chi \in \FF$.
	
	Addends of LHS and RHS are encoded into polynomials. Hence, applying sumcheck to show both sides equal to a common $T \in \FF$.
\end{frame}